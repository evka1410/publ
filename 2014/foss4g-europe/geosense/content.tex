\begin{frame}\begin{center}

\includegraphics[width=0.4\textwidth]<1>{../../../common/imgs/OSGeo_logo.png}
\hfill
\includegraphics[width=0.4\textwidth]<1>{../../../common/imgs/osgeo-cz.png}

\includegraphics[width=0.6\textwidth]<2>{../../../common/imgs/map-coding.png}

\note<1>{I'm the member of OSGeo's board of directors
Foundation and chair of czech local chapter, we call ourselves "Open
GeoInfrastracture"}

\note<2>{
More then ten years, I'm involved in the development of open source software for
geospatial, some of you may remember me as member of GRASS development team and
founder of PyWPS project. But since 2007, so about 7 years, I'm focused mostly
on development of geo applications for web.}

\end{center}\end{frame}

%%%%%%%%%%%%%%%%%%%%%%%%%%%%%%%%%%%%%%%%%%%%%%%%%%%%%%%%%%%%%%%%%%%%%%%%%%
\begin{frame}\begin{center}

\includegraphics[width=\textwidth]<1>{imgs/grass-openlayers.png}

\note<1>{I switched from desktop to web developemnt during 2007, and it was
OpenLayers library, which tought me, how to write clean and nice looking code
(at least compared to GRASS GIS) in JavaScript. I, unlike some of you, never
feard OpenLayers complexity, since it always was so clean and simple, yet
powerfull, at least compared to what I was used to in the C-world from 80's - I
mean GRASS again. 
}

\includegraphics[width=\textwidth]<2>{imgs/distros-browsers.png}

\note<2>{In other words, I replaced issues with various libraries verions and
    linux distributions, byt issues caused by various browsers and their
    versions}

\includegraphics[width=\textwidth]<3>{imgs/swissarmyknife-ol.jpg}

\note<3>{
OpenLayers 2 was and still is very inspiring to me also regarding it's features.
It is true geospatial library, with capabilities to display all data you needed,
parse all formats you came across and to mock any type of functionality within
day, your boss required.
}
\end{center}\end{frame}

%%%%%%%%%%%%%%%%%%%%%%%%%%%%%%%%%%%%%%%%%%%%%%%%%%%%%%%%%%%%%%%%%%%%%%%%%%
\begin{frame}\begin{center}

\includegraphics[width=0.8\textwidth]<1>{imgs/ol-vectors.png}

\note<1>{
OpenLayers is great in displaying vector data too. But howmany times did this
happen to you, as to me?
}

\includegraphics[width=0.8\textwidth]<2>{imgs/ol-vectors-stuck.png}
\note<2>{
We've crached the limits of the web around the first decate
of year 2000. SVG was supported on most browsers, except for IE. Even IE 8 was
already around, we still needed to support IE 6 for some gov. institutsions.
Combinatiton of SVG and VML was working well, but it was so terribly slow, and
displaying larger datasets (speaking about size of several MBs) was like
impossible.
}

\includegraphics[width=0.6\textwidth]<3>{imgs/leaflet-ol-ol3.png}

\note<3>{
Last years, I had a chance to visit FOSS4G in Nottingham and I wanted to make
research, wether we stick with OpenLayers 2, go to Leaflet or move on to
OpenLayers 3. My personal feeling around that time was, we stick with OpenLayers
2 and will observe, how the situation will develop.
}
\end{center}\end{frame}

%%%%%%%%%%%%%%%%%%%%%%%%%%%%%%%%%%%%%%%%%%%%%%%%%%%%%%%%%%%%%%%%%%%%%%%%%%
\begin{frame}\begin{center}

\includegraphics[width=\textwidth]<1->{imgs/ol2tool3.png}

\note<1>{But after great session, full of OpenLayers 3 presentations, given by
Tim Schaub, Eric Lemoine, Tom Payne and others, I was keen on trying OpenLayers
3 on real project, it was time to move on.}

\note<2>{When I came back home, we started to rewrite old, OpenLayers 2-based
geoportal from scratch}
\end{center}\end{frame}

%%%%%%%%%%%%%%%%%%%%%%%%%%%%%%%%%%%%%%%%%%%%%%%%%%%%%%%%%%%%%%%%%%%%%%%%%%
\begin{frame}\begin{center}

\includegraphics[width=\textwidth]<1->{imgs/geoportal.png}

\note<1>{The old geoportal, how we call it nowadays, is still OpenLayers 2-based
beast, with all good and bad, what belongs to the passed era:}

\note<2>{
\begin{itemize}
    \item It's big, speaking about size of javascript files, you need to
    transfare from server to client
    \item It's almost impossible to render larger amount of data (even with
    canvas renderer it's much easier)
    \item It's hard to maintain, because of  not so well structured code
    \item We simply dislkied to continue in this way
\end{itemize}
}
\end{center}\end{frame}

%%%%%%%%%%%%%%%%%%%%%%%%%%%%%%%%%%%%%%%%%%%%%%%%%%%%%%%%%%%%%%%%%%%%%%%%%%
\begin{frame}\begin{center}

\includegraphics[width=\textwidth]<1>{imgs/complicated.jpg}

\note<1>{
We started with OpenLayers 3. It took us two weeks to setup and think about the
whole dev environemnt, since OpenLayers 3 is relaying strongly on Google Closure
library. We experimented with Plovr - Java building tool for closure - for
couple of months, till we got rid of it during last month, we are using
bootstrap for UI desing, but withou JQuery -- where ever we should refer to
JQuery, we stick with standard closure elements.
}

\only<2>{
\begin{itemize}
    \item old portal re-implemented
    \item vector-data focused
    \item tests (CasperJS)
    \item 250kB
    \item strong type control
\end{itemize}
}

\note<2>{
Nowadays we have 
\begin{itemize}
    \item 90\% of functionality of the old portal reimplemented
    \item all vector data are rendered as vector data, which will enable us more
    closer user integration
    \item test driven development using casprejs
    \item small compiled library
    \item system, which is hammering us over figter tips, once we do something nasty
    during the development
\end{itemize}
}
\end{center}\end{frame}

%%%%%%%%%%%%%%%%%%%%%%%%%%%%%%%%%%%%%%%%%%%%%%%%%%%%%%%%%%%%%%%%%%%%%%%%%%
\begin{frame}\begin{center}

\only<1>{
    map vs. data
}

\note<1>{
The portal itself has two modes - map-centrique and data-centrique.}

\note<2>{
In the
map-centrique mode, user can browse the map, switching layers and so on. But
after the user opens data table
}

\includegraphics[width=\textwidth]<2>{imgs/geoportal2.png}

\note<3>{
The map becames just supporting overview map to the database application, where
user can filter, search and sort all features, available in the dataset.
}

\includegraphics[width=\textwidth]<3>{imgs/geoportal2-table.png}

\note<3>{
Other functions, such as feature detail, measuring or filtering are available as
well.
}
\end{center}\end{frame}

%%%%%%%%%%%%%%%%%%%%%%%%%%%%%%%%%%%%%%%%%%%%%%%%%%%%%%%%%%%%%%%%%%%%%%%%%%
\begin{frame}\begin{center}

\includegraphics[width=\textwidth]<1>{imgs/burn-fat.jpg}

\note<1>{
The whole application is now able to handle 10~000 (ten thounsends) of features,
with just 300KB of javascript code. It loads faster, scales nicely, we like to
work with it.
}

\includegraphics[width=\textwidth]<2>{imgs/dirty-tablet.jpg}

\note<2>{Since we are using boostrap, we are focusing ourselves on tablet
version as well. The whole portal is wokring wihout mouse or keaboard, using
only biggger touch device}

\only<3>{
    API
}

\note<3>{
The portal is intended to work with proprietary server information system API,
but it's written in a way, that it should be possible to use it in separate way.
}
\end{center}\end{frame}

%%%%%%%%%%%%%%%%%%%%%%%%%%%%%%%%%%%%%%%%%%%%%%%%%%%%%%%%%%%%%%%%%%%%%%%%%%
\begin{frame}\begin{center}
\note<1>{
    Problems we had to face:
    \begin{itemize}
        \item It is very complicated for non-javascript programmers, to setup
            and think over development environemnt. The whole world of
            JavaScript is moving forward too fast, to catch it up. There is
            completely new world with magical words, like PhantomJS, NodeJS,
            CaspreJS, closure.
        \item The environment is not settelt yet, new JavaScript build system
            are creatd every day. Yesterday everybody used Grunt, nowadays, Gulp
            is the big hype. What you really need? IMHO python-based build
            script will do.
        \item We tryed to use Gulp for building, but result was, that NodeJS was
            not able to handle all 500 files, we use for our geoportal.
        \item Also gui libraries are changing fast. Jquery is no longer
            must-use library, not even for mobile. Bootstrap seems to be one of
            the most promissing. But using bootstrap without jquery means, you
            have to deal with some fancy gui elements. We are now implementing
            everything using standard closure elements.
        \item And not to forget templating systems. Closure has one, but it's
            not so nice. Now, when react is around, we would like to switch to
            it
    \end{itemize}
}
\note<2>{
    What do I want to say: the javascript world is chaning fast, there is
    plently of building systems (even you can scratch new one for your project),
    there are number of libraries, which can be used for your not-only-mapping
    applications, there are several gui libraries and templating systems. All
    together it's somehow very complicated for originaly server-side coders. But
    we all enjoy this new world of JavaScript, it gives us some time to play
    again.
}
\end{center}\end{frame}

%%%%%%%%%%%%%%%%%%%%%%%%%%%%%%%%%%%%%%%%%%%%%%%%%%%%%%%%%%%%%%%%%%%%%%%%%%
\begin{frame}\begin{center}

\note<1>{
Conclusion to us: if you are waiting with openlayers 3, go for it. It was hard
at the beginning to keep our steps synchronized with OpenLayers team, but it was
possible. Today's changes are relatively small and we did not break
compatibility for long time already.
}

\note<2>{
OpenLayer 3 is nice library, which I personally think is one step ahead of
Leaflet. Do not get me wrong, Leaflet is very nice and powerful tool, but we
need the complexity of OpenLayers and so, we consider OpenLayers 3 as good
choice.
}
\end{center}\end{frame}

%%%%%%%%%%%%%%%%%%%%%%%%%%%%%%%%%%%%%%%%%%%%%%%%%%%%%%%%%%%%%%%%%%%%%%%%%%
\begin{frame}\begin{center}
?
\end{center}\end{frame}
