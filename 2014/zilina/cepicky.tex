\documentclass[twoside]{oss-conf}
\usepackage[english,czech]{babel}             % use english and your language respectively ... czech, polish, ...
\usepackage[T1]{fontenc}
\usepackage[utf8]{inputenc}                    % use only UTF-8 encoding 
\usepackage{amsmath}
\usepackage{amscd,amssymb,amsfonts}
\usepackage[sort,nocompress]{cite} 
\usepackage{graphicx}                          % use for include pictures *.jpg, *.png, *.pdf
  %\usepackage{color}
  %\usepackage{colortbl}
  %\usepackage{multicol}
  %\usepackage{comment}
\usepackage{fancyhdr}
\usepackage[pdftex,unicode,bookmarks=false]{hyperref}  % for e-mail and url adress
\usepackage{eurosym}
\usepackage{listings}



%.............................................................................
\begin{document}

%..................................................................................
\setcounter{page}{1}                            % číslo prvej stránky príspevku
\def\volumeDOI{OSSConf 2014:\  }                % číslo zborníka
\def\konfera{Konferencia OSSConf 2014}

\pagestyle{fancy}
\fancyfoot{}\fancyhead[LE,RO]{\thepage}\fancyhead[LO]{\nouppercase{\leftmark: \rightmark}}\fancyhead[RE]{\nouppercase{\konfera}}



%........................................................................................................................................
%     TU ZAČÍNA PÍSAŤ AUTOR
%........................................................................................................................................


	

%.............................................. pre slovenčinu
%\def\solutionname{Riešenie}
%\def\addressS{Kontaktná adresa}
%\def\addressSis{Kontaktné adresy}
%\def\rbCurrentaddress{Aktuálna adresa}
%\def\rbEmailaddress{E-mailová adresa}
%\def\keywordsnameS{Kľúčové slová}     
%.............................................. pre češtinu odpoznámkovať
\def\solutionname{Řešení}
\def\addressS{Kontaktní adresa}
\def\addressSis{Kontaktní adresy}
\def\rbCurrentaddress{Aktuální adresa}
\def\rbEmailaddress{E-mailová adresa}
\def\keywordsnameS{Klíčová slova}


%%.............................................. slovenský alebo český názov
% pre kratšie názvy  použi iba príkaz   \title{CELÝ NÁZOV ČLÁNKU}  
\title[Otevřený Geosvět]{Otevřený svět Geo*} 


%.............................................. anglický názov
\titleA[]{Opened world of Geo*}


%.............................................. autori (jeden alebo viac) .. každý podľa vzoru:
% \author[Skratka mena do hlavičky]{Pred titul}{Meno priezvisko}{Po titul}{skratka štátu}
% \address{adresa..práca}
% \curraddress{alternatívna adresa}
% \emailh{sanonymous@water.sun.xy}
% \urladdressh{http://www.sanous.tows.xy/~sanonymous}



\author[J. Čepický]{Jáchym Čepický}{Ing.}{}{}
\address{Česká republika}
\curraddress{Česká republika}
\emailh{jachym@les-ejk.cz}
\urladdressh{http://les-ejk.cz, http://gismentors.eu}

%.............................................. venovanie
%\dedicatory{Ak to chcem niekomu venovať\dots nepovinné}


%.............................................. kľúčové slová slovensky/anglicky
\keywords{GIS, Open Source, Free Software}
\keywordsA{GIS, Open Source, Free Software}

%.............................................. (obligatory) AMS Classification 2000
% The 1-st classification is obligatory, the 2-nd classification is optional
% \subjclass{primary}{secondary}      f.e. \subjclass{35R35, 49N50}{}
  % \subjclass{35R35, 49M15, 49N50}{}



%.............................................. slovenský abstrakt
\selectlanguage{czech}

\begin{abstract}
Open Source Geosvět se od toho ,,normálního`` IT svět trochu v něčem liší. Co je
to ale otevřenost, jako koncept, kde se vzala tak najednou? A nebo to nebylo
najednou? A proč vlastně veškerý software není otevřený? A měl by být?
\end{abstract}


%.............................................. anglický abstrakt
\selectlanguage{english}

\begin{abstractA}
    Open Source for geospatial is always little bit different from ,,standard``
    IT world. What is it \emph{to be open}, where does it come from? Or it was
    not all at sudden? Why isn't all software open sourced? Should it be?
\end{abstractA}



\selectlanguage{czech}
%.............................................. moje makrá
\newtheorem{theorem}{Veta}[section]
\newtheorem{corollary}[theorem]{Tvrdenie}
\newtheorem{lemma}[theorem]{Lema}
\newtheorem{exmple}[theorem]{Príklad}
\newtheorem{defn}[theorem]{Definícia}
\newtheorem{rmrk}[theorem]{Poznámka}
            %%% for no-italic, numbered environments, use:
\newenvironment{definition}{\begin{defn}\normalfont}{\end{defn}}
\newenvironment{remark}{\begin{rmrk}\normalfont}{\end{rmrk}}
\newenvironment{example}{\begin{exmple}\normalfont}{\end{exmple}}
            %%% for unnumbered environments, use f.e.
\newtheorem*{remarque}{Poznámka}

\newcommand\dd{\mathop{\rm d\!}\nolimits}
\newcommand\sgn{\mathop{\rm sgn}\nolimits}



%.............................................. telo článku
\maketitle
\bigskip

Více než deset let se zabývám vývojem open source software
pro geoinformatiku. Úzce s tím souvisí i to, že trochu déle se počítám mezi
uživatele operačního systému GNU/Linux. To obyčejně nepovažuji za důležité
zmiňovat, ale v kontextu této přednášky to považuji za významnou skutečnost.

Během studií jsem u svého kamaráda našel v jeho knihovničce knížku Učíme se Red Hat Linux
a při jejím čtení mi došlo, že vedle počítačů jak je znám existuje jiný
svět, který bych rád poznal. V té době pro mě počítač bylo PC s MS Windows 98,
používal jsem Excel na tvorbu protokolů a prací pro svůj obor (kterým bylo lesní
inženýrství). Nikdy jsem neměl ambici programovat, počítače jsem chápal jako
nástroje, které by mi měli v ideálním případě usnadnit práci. V knížce jsem se
seznámil se základními ideemi open source a free software a byly mi sympatické,
jakkoliv jsem nedokázal pochopit, proč by někdo sdílel svoji práci --
intelektuální vlastnictví -- s kýmkoliv jiným. Když se mi v krátké době podařilo
pořídit si svůj vlastní nový počítač, požádal jsem kamaráda o pomoc
nainstalovali jsme můj první Linux. 

Od té doby jsem se stal uživatelem open source a free software a postupem času i
jeho spolutvůrcem. Naučil jsem se, jak funguje komunita uživatelů a přispěvatelů
do open source software, pochopil jsem některé jemné rozdíly mezi svobodným
software a otevřeným software a tak dále. A dnes chápu, že open source, nebo
obecně tak zvaný crowd source -- sdílený přístup -- není žádná anomálie.

Většina z vás už určitě slyšela o tom, že hnutí free software založil Richard
Matthew Stallman v 80.~letech minulého století prací na operačním systému GNU.
Definoval tři základní svobody nebo práva, které by měl každý uživatel software
mít: právo vidět zdrojový kód, právo sdílet zdrojový kód a právo měnit zdrojový
kód a dále ho sdílet.

V systém se stal úspěšný také díky jádru operačního systému Linux. Dnes je Linux
nejpoužívanějším serverovým operačním systémem, používá se v mobilních
zařízeních, na super počítačích. Proč se tak stalo? Protože je to otevřený systém
-- systém, ve kterém ostatní sdílí svůj kód -- svoje know-how -- svoje
 intelektuální vlastnictví s ostatními.

Ale koncept sdílení není přece vynalezen v 80 letech. Západní věda tak jak ji
chápeme je snad od dob antiky postavena na stejných principech: publikuji
výsledek způsobem, že je opakovatelný, očekávám, že někdo jiný mou práci
zopakuje, přezkouší její platnost, případně navrhne některé změny, a výsledek
opět publikuje. Vědecká komunita díky sdílení informací posouvá hranice našeho
poznání dále, stejně jako komunita vývojářská posouvá hranice software.

Open source software není jediný příklad pro fungující intelektuální
spoluvlastnictví. Nikdo snad nepochybuje, že wikipedia je spolehlivý informační
zdroj a přitom na její obsah nemá nikdo monopol, neexistuje globální autorita,
která by strážila věcný obsah Wikipedie. OpenStreetMap je podle mě další
úspěšný projekt. Jako geoprostoroví profesionálové můžete zpochybňovat polohovou
přesnost dat, jejich faktickou správnost nebo aktuálnost, ale nemůžete
zpochybnit, že globálně vzato je to ucelený dataset který nemá v prorietárním
světe obdoby.

Všechny tyto příklady mají díky svým licencím společné tři základní vlastnosti,
které vyzývají další a další uživatele aby se přidali: vidět obsah, sdílet obsah
a provádět změny v obsahu a sdílet tyto změny dál.

Je jen málo oblastí geoinformatiky, které by nebyly pokryty kvalitním open
source software -- i když připouštím, že jsou. Stále vznikají nové programy pro
další oblasti, staré ozkoušené projekty jsou ale stále zde, stabilní, s
rozšířenou vývojářskou a uživatelskou komunitou, stále dostávají nové funkce, ty
staré jsou oprašovány a udržovány.

Dovolte mi jako příklad zmínit několik z nich:
\begin{description}
    \item[GRASS GIS] -- Desktopový GIS, vyvíjený původně americkou armádou od roku 1982,
který obsahuje množství kódu českých a slovenských vývojářů.

\item[QGIS], který je sice mladší než GRASS, ale na Slovensku je opět velice populární.
Daří se mu stále více nahrazovat co do vzhledu a funkcí komerčně rozšířený
ArcGIS.

\item[PostGIS] je prostorové rozšíření databáze PostgreSQL.

    \item[MapServer a GeoServer] jsou zástupci serverových programů, které po světě
vydávají mapy a jsou nasazováni i všude tam, kde je potřeba řešit iniciativu
INSPIRE.

\item[PyCSW] je serverová aplikace shromažďující a vydávající metadata.
\end{description}

A mnoho dalších.

Kdo jste ještě nebyli na žádné konferenci zabývající se obecně geo* open source
software, doporučuji vám navštívit některou z konferencí FOSS4G, ať už letos ve
Spojených státech globální, nebo její evropskou odnož v Brémách. Máte také
možnost nahlídnout přímo do  vývojářské kuchyně na některém z code sprints,
naposledy ve Vídni, další možnost budete mít v italském Bozlanu. 

Z toho všeho co říkám si možná říkáte: proč to tedy nikdo nepoužívá, když je to
tak skvělý produkt? Proč není open source software dávno rozšířenější, než jak
to vypadá?

Důvody pro tento stav -- a je jedno jestli je to objektivní fakt nebo subjektivní
pocit -- jsou samozřejmě mnohé a jak už to bývá, mají povahu vnitřní i vnější.

Jako jeden z prvních důvodů (bez nároku na prvenství z hlediska významnosti)
uvedu to, že člověk používá to, co zná. Věřím, že obsahem výuky na
školách ve všech oborech a stupních by měly být především obecně platné
principy, až od nich odvozené konkrétní situace. Učíme obecně Archimedův zákon,
a teprve následně několik jeho praktických aplikací, jako že ve vodě je slon
lehčí nebo spolu s inženýry španělského námořnictva, že ponorka, má-li se vynořit,
musí mít především v součtu menší objemovou hustotu, než voda. Proč se ale tento
princip uplatňuje při výuce software jen velice zřídka? Proč výuka počítačí
obecně a GIS konkrétně se téměř bez výjimky provádí na konkrétním jednom
software?

Dalším důvodem je neexistence jednotného telefonního čísla -- chybí marketingové
oddělení open source GIS, není tu někdo, kdo by zákazníkům vysvětlil proč je
konkrétní produkt ten nejlepší (v absolutním i relativním významu). U open source
software se předpokládá, že jste dospělí, že jste nebo se chcete stát experty a
proto na to, co je pro váš případ to nejlepší si přijdete sami. Open source je
náročný na lidské zdroje, pokud nemáte u sebe někoho dalšího, jste v tom tak
trochu sami, ,,jenom`` s podporou komunity. To může hodně lidí odradit, ale
bylo mnohokrát  ilustrováno, že podpora komunity funguje, první odpověď v mailing
listech bývá u větších projektů v řádu minut.

Dalším důvodem bude roztříštěná nabídka open source software. Jak jsem řekl
dříve, nemáte se moc koho zeptat a ještě ke všemu si musíte vybírat z množství
variant -- sám s tím mám často problém. Provozní výzkum, porovnávání různých
programů a sledování nových je denní chléb. Mám vzít TileCache, MapStach,
GeoServer cache, MapCache? A co je to ten Mapnik? A jaký je rozdíl mezi Leaflet
a OpenLayers? Na tyto otázky získáte nejlépe odpověď tak, že budete číst nebo se
zeptáte někoho, kdo to ví, ale jak jsem již řekl -- ve vašem okolí je většinou
problém najít někoho kdo by to věděl. Nebo snad ne?

Dalším důvodem je nejednotný systém návazného vzdělávání. Existují spíše
jednotlivci, nabízející školení. Školení jsou ale nejednotná, různá obsahem i
kvalitou.

Důvodů, proč open source software není úspěšnější -- myšleno absolutně -- je ještě
celá další řada. Některé z nich a některé z již zmíněných se bude snažit zaplnit
na nejen naší geo- scéně projekt GISMentors.

Na Slovensku už jsou firmy, nabízející, podle toho, co jsem mohl vidět, open
source řešení na profesionální úrovni, včetně možnosti, jak říkávám ,,sáhnout do
stroje`` a opravit chybu nebo přidat funkci. Myslím, že i vzhledem k tomu kolik
Slovenských vývojářů se pohybuje na open source GIS scéně, má Slovensko světlou
geo-budoucnost.


\begin{acknowledgment}[Poďakovanie]
Tento príspevok vznikol s láskavým prispením grantu VEGA-FRC~2012/003 štedro sponzorovaného
Slovenskou grantovou agentúrou.
\end{acknowledgment}


\end{document}
