% $Header: /home/vedranm/bitbucket/beamer/solutions/conference-talks/conference-ornate-20min.en.tex,v 90e850259b8b 2007/01/28 20:48:30 tantau $

\documentclass[xcolor=dvipsnames]{beamer} 


% This file is a solution template for:

% - Talk at a conference/colloquium.
% - Talk length is about 20min.
% - Style is ornate.


% Copyright 2004 by Till Tantau <tantau@users.sourceforge.net>.
%
% In principle, this file can be redistributed and/or modified under
% the terms of the GNU Public License, version 2.
%
% However, this file is supposed to be a template to be modified
% for your own needs. For this reason, if you use this file as a
% template and not specifically distribute it as part of a another
% package/program, I grant the extra permission to freely copy and
% modify this file as you see fit and even to delete this copyright
% notice. 


\mode<presentation>
{
  %\usetheme{boxes}
  %\usecolortheme{seagull}
  % or ...

  \setbeamercovered{transparent}
  % or whatever (possibly just delete it)

    \usecolortheme[named=OliveGreen]{structure} 
    \usetheme[height=7mm]{Rochester} 
    \setbeamertemplate{items}[ball] 
    \setbeamertemplate{blocks}[rounded][shadow=true]
}


\usepackage[czech]{babel}
% or whatever

\usepackage[utf8]{inputenc}
% or whatever

\usepackage{hyperref}
%\definecolor{links}{HTML}{2A1B81}
\hypersetup{colorlinks,linkcolor=OliveGreen,urlcolor=OliveGreen}

\usepackage{times}
\usepackage[T1]{fontenc}
% Or whatever. Note that the encoding and the font should match. If T1
% does not look nice, try deleting the line with the fontenc.


\title[HSLayers] % (optional, use only with long paper titles)
{HSLayers map viewer}

\subtitle {Do you want to INSPIRE?}

\author[J. Čepický] % (optional, use only with lots of authors)
{Jáchym~Čepický\inst{1}, Štěpán~Kafka\inst{1}, Přemysl~Vohnout\inst{2}}
% - Give the names in the same order as the appear in the paper.
% - Use the \inst{?} command only if the authors have different
%   affiliation.

\institute % (optional, but mostly needed)
{
  \inst{1}%
  Help Service - Remote Sensing s.r.o. \\
  Benešov\\
  \url{http://hsrs.cz}\\

  \inst{2}%
  Czech Center for Science and Society\\
  Prague\\
  \url{http://ccss.cz}\\
}
  
% - Use the \inst command only if there are several affiliations.
% - Keep it simple, no one is interested in your street address.

\date[] % (optional, should be abbreviation of conference name)
{FOSS4G-CEE 2013, Bucure\c{s}ti}
% - Either use conference name or its abbreviation.
% - Not really informative to the audience, more for people (including
%   yourself) who are reading the slides online

% If you have a file called "university-logo-filename.xxx", where xxx
% is a graphic format that can be processed by latex or pdflatex,
% resp., then you can add a logo as follows:

\pgfdeclareimage[height=1.cm]{conference-logo}{imgs/conference-logo.png}
\pgfdeclareimage[height=.5cm]{hsrs-logo}{imgs/hsrs.png}
\pgfdeclareimage[height=1.0cm]{osgeo-logo}{imgs/osgeo.png}
\pgfdeclareimage[height=.5cm]{hslayers-logo}{imgs/hslayers.png}
%\pgfdeclareimage[height=2.0cm]{-logo}{university-logo-filename}
\logo{\pgfuseimage{conference-logo}}



% Delete this, if you do not want the table of contents to pop up at
% the beginning of each subsection:
\AtBeginSubsection[]
{
\logo{\pgfuseimage{conference-logo}}
  \begin{frame}<beamer>{Content}
    \tableofcontents[currentsection,currentsubsection]
  \end{frame}
}


% If you wish to uncover everything in a step-wise fashion, uncomment
% the following command: 

%\beamerdefaultoverlayspecification{<+->}


\begin{document}

\begin{frame}
  \titlepage
\end{frame}

\begin{frame}{TOC}
  \tableofcontents
  % You might wish to add the option [pausesections]
\end{frame}

\section{INSPIRE}

\begin{frame}{SDI}
Spatial data infrastructure\footnote{\url{http://en.wikipedia.org/wiki/Spatial_data_infrastructure}}
\begin{itemize} 
    \item software client - to display, query, and analyse spatial data (this could be a browser or a Desktop GIS),
    \item a catalogue service - for the discovery, browsing, and querying of metadata or spatial services, spatial datasets and other resources,
    \item a spatial data service - allowing the delivery of the data via the Internet, processing services - such as datum and projection transformations,
    \item a (spatial) data repository - to store data, e.g. a Spatial database, GIS software (client or desktop) - to create and update spatial data
\end{itemize}
\end{frame}

\begin{frame}{INSPIRE}
Infrastructure for Spatial Information in the European Community
\begin{itemize} 
    \item 2007
    \item General SDI framework for European purposes
    \item Implementing rules 
        \begin{itemize}
            \item Metadata
            \item Network services,
            \item Data and Service Sharing
            \item Monitoring and Reporting
        \end{itemize}
\end{itemize}
\end{frame}

\begin{frame}
\begin{block}{INSPIRE Geoportal}
Inspire geo-portal means an Internet site, or equivalent,
providing access to the services referred to in Article 11(1);
\end{block}
\begin{itemize}
    \item Discovery
    \item \alert{View}
    \item Download
    \item Transform
    \item Invoke
\end{itemize}
\end{frame}

\begin{frame}
\begin{block}{INSPIRE View service}
view services making it possible, as a minimum, to
\end{block}
\begin{itemize}
    \item display
    \item navigate
    \item zoom in/out
    \item pan
    \item overlay viewable spatial datasets
    \item display legend
    \item display relevant content of metadata
\end{itemize}
\end{frame}

%%%%%%%%%%%%%%%%%%%%%%%%%%%%%%%%%%%%%%%%%%%%%%%%%%%%%%%%%%%%%%%%%%%%%%%
\section{HSLayers}
\begin{frame}
\begin{block}{HSLayers}
HSLayers is yet another JavaScript Mapping framework. You can use it for building rich mapping portals, as well as use it’s parts for improving your simple OpenLayers-based map.
\end{block}
\end{frame}

\begin{frame}{HSLayers structure}
\begin{itemize}
    \item Components based on OpenLayers
    \item Components based on ExtJS
    \item Patches for OpenLayers
\end{itemize}
\end{frame}

\begin{frame}{Why OpenLayers?}
\begin{center}
    Why OpenLayers?!
\end{center}
\end{frame}

\begin{frame}{Why ExtJS?}
\begin{center}
    Why ExtJS?!
\end{center}
\end{frame}

\begin{frame}{Why not GeoExt?}
\begin{center}
    Why Not GeoExt?!
\end{center}
\end{frame}

\begin{frame}{OpenLayers components}
\scriptsize
\begin{columns}
\column{.45\textwidth}
\begin{itemize}
    \item HSLayers.Map (HSLayers.Metadata, HSLayers.User, \dots)
    \item HSLayers.Popup
    \item HSLayers.Control.Attribution
    \item HSL.Control.BoxLayerSwitcher
    \item HSLayers.Control.Editing
    \item HSLayers.Control.Measure
    \item HSLayers.Control.PanZoomBar
    \item HSLayers.Control.Query
\end{itemize} 
\column{.45\textwidth}
\begin{itemize}
    \item HSLayers.Control.ScaleSwitcher
    \item HSLayers.Control.State
    \item HSLayers.Layer.OWS, WMS, WFS, WCS
    \item HSLayers.Layer.TreeLayer
    \item HSLayers.Layer.MultiLayer
    \item HSLayers.Format.PrintContext
    \item HSLayers.Format.State
    \item HSLayers.Format.KML
    \item HSLayers.Format.WMC
\end{itemize}
\end{columns}
\end{frame}

\begin{frame}{ExtJS Components}
\begin{itemize}
    \item HSLayers.LayerSwitcher
    \item Context manager
    \item Printig client
    \item Permalink window
    \item MapPortal -- integrating all together
    \item \dots
\end{itemize}
\end{frame}

%%%%%%%%%%%%%%%%%%%%%%%%%%%%%%%%%%%%%%%%%%%%%%%%%%%%%%%%%%%%%%%%%%%%%%%
\begin{frame}{Server components}
\begin{itemize}
    \item Ajax proxy (also POST and other request types)
    \item StatusManager -- PHP script for storing sessions and permalinks
    \item Printing module
    \hrule
    \item Proxy4ows -- transform WFS and WCS services into WMS (and warp WMS, if
    needed)
    \item Editing
\end{itemize}
\end{frame}

\section*{Conclusion}
\begin{frame}
\begin{block}{Geoportals}
Map portals do not work 
\end{block}
Brian Timoney, \url{http://mapbrief.com}
\end{frame}

\section*{Conclusion}
\begin{frame}
\begin{block}{Geoportals}
Map portals do work, we are just doing it wrong
\end{block}
Chris Helm
\end{frame}


\begin{frame}
    \begin{center}
        Jáchym Čepický \\
        jachym@hsrs.cz \\
        \url{http://bnhelp.cz/} \\
        \url{http://hslayers.org}
    \end{center}
\end{frame}

\end{document}


